\documentclass[12pt]{article}
\usepackage[utf8]{inputenc}
\usepackage[colorlinks=true,linkcolor=blue]{hyperref}
\usepackage{amsmath, amssymb}
\usepackage{enumitem}


\title{CISC203 notes\\Discrete Structures II}
\author{Andrew Aquino}
\date{}

\begin{document}

\maketitle

\section*{Introduction}

To whom this may concern, these are some notes for CISC203, 
a course I took in the fall of 2025. 
I am writing these notes for my own benefit, 
and I make no claims about their accuracy or completeness. 
Use at your own risk.\\

Check me out at \url{https://github.com/andrewSmellz}

\tableofcontents
\newpage

\section{Group Theory}

\subsection{Division}

\noindent\textbf{Definition 1:} ($a \mid b$) 
\label{def:divides}
\\Let $a, b \in \mathbb{Z}$, with $a \neq 0$.  
If $b = ak$ for some $k \in \mathbb{Z}$, then we say that $a$ divides $b$,  
or that $a$ is a divisor of $b$. 
\[\text{This is denoted as: } a \mid b\]

\vspace{5mm}

\noindent\textbf{Theorem 2:} (Division Algorithm) 
\label{thm:division}
\\Let $a, b \in \mathbb{Z}$, with $b > 0$.  
There exists a unique pair of integers $q$ and $r$ such that:

\[
a = qb + r \quad \text{where} \quad 0 \leq r < b
\]

\noindent This expresses $a$ as a multiple of $b$ plus a remainder $r$.
Additionally, we call $d$ the divisor, $a$ the dividend, $q$ the quotient, and $r$ the remainder.

\vspace{5mm}

\noindent\textbf{Definition 3:} (Division and Modulus)
\label{def:divmod}
\\Let $a, b \in \mathbb{Z}$, with $b > 0$. 
Then we define the division and modulus operations as follows:

\[
q = a \div b, \quad r = a \bmod b
\]

where $q$ and $r$ are the unique pair of numbers (by Theorem \hyperref[thm:division]{2}) where $a = qb + r$ and $0 \leq r < b$.

\vspace{5mm}

\noindent\textbf{Definition 4:} (Congruence Modulo $n$)
\label{def:congruence}
\\Let $x,y,n \in \mathbb{Z}$, with $n > 0$. If $n \mid (x - y)$, we can say that $x$ and $y$ are congruent modulo $n$. 
This is denoted as: $x \equiv y \pmod{n}$.
The set of all integers congruent to an integer $a$ modulo $n$ is called the congruence class of $a$ modulo $n$.

\vspace{5mm}

\noindent\textbf{Example 5:} 
\label{ex:modexample}
\\ $53 \equiv 23 \pmod{10}$ means that $53-23=30$ is a multiple of 10. 
\\However, $53 \bmod 10 = 3$ and $23 \bmod 10 = 3$, meaning that the remainder of $53 \div 10$ is $3$.

\vspace{5mm}

\noindent\textbf{Theorem 6:} 
\label{thm:modiff}
\\Let $a,b,n \in \mathbb{Z}$, with $n > 0$. 
Then 
\[
a \equiv b \pmod{n} \iff a \bmod n = b \bmod n
\]

\vspace{5mm}

\noindent\textbf{Example 7:}
\label{ex:modexample2}
\[
9 \equiv 17 \pmod{4} \text{ is true, so we also have } 
9\bmod 4 = 1 \text{ and } 17 \bmod 4 = 1
\]

\vspace{15mm}

\subsection{Greatest Common Divisor}

\noindent\textbf{Definition 8:} (Common Divisor)
\label{def:commondivisor}
\\Let $a,b \in \mathbb{Z}$. If an integer $d$ divides both $a$ and $b$, we say that $d$ is a \textbf{common divisor} of $a$ and $b$.

\vspace{5mm}

\noindent\textbf{Example 9:}
\label{ex:commondvs}
\\The common divisors of 12 and 18 are $\pm1, \pm2, \pm3, \text{ and } \pm6$.
\\The common divisors of 25 and 50 are $\pm1, \pm5, \pm10, \text{ and } \pm25$.

\vspace{5mm}

\noindent\textbf{Definition 10:} (Greatest Common Divisor)
\label{def:gcd}
\\Let $a,b \in \mathbb{Z}$. We say that an integer $d$ is the greatest common divisor of $a$ and $b$, provided that:
\begin{enumerate}
    \item $d$ is a common divisor of $a$ and $b$
    \item If $e \mid a$ and $e \mid b$, then $e \leq d$
\end{enumerate}
The greatest common divisor of $a$ and $b$ is denoted as $\gcd(a,b)$. By definition it is always positive.

\vspace{5mm}

\noindent\textbf{Example 11:}
\label{ex:gcdnaive}
\\The greatest common divisor of 18 and 12 is 6. Using the naive method: 
\begin{enumerate}
    \item Find all divisors of $a$ 
    \item Find all divisors of $b$
    \item Choose the largest number that is a divisor of both $a$ and $b$
\end{enumerate}

\vspace{5mm}

\noindent\textbf{Theorem 12:} (Fundamental Theorem of Arithmetic)
\label{thm:fta}
\\Every integer greater than 1 can be written uniquely as a prime or as the product of primes, written in nondecreasing order.

\vspace{5mm}

\noindent\textbf{Example 13:}
\label{ex:primefact}
\\the prime factorizations of 20,23,288, and 621 are:
\[
r_0 = 20 = 2 \cdot 2 \cdot 5 = 2^2 \cdot 5, \]
\[
r_1 = 23 = 23, \]
\[
r_2 = 288 = 2 \cdot 2 \cdot 2 \cdot 2 \cdot 2 \cdot 3 \cdot 3 = 2^5 \cdot 3^2, \]
\[
r_3 = 621 = 3 \cdot 3 \cdot 3 \cdot 23 = 3^3 \cdot 23
\]
We can find the greatest common divisor of two numbers by taking the product of all common prime factors. That is, 
\[
\gcd(r_0, r_2) = 2^2 = 4 \quad \text{and} \quad \gcd(r_2, r_3) = 3^2 = 9.
\]
Since \(r_0\) does not have any common prime factors with \(r_3\), we have
\[
\gcd(r_0, r_3) = 1.
\]

We can also find the least common multiple of two numbers using this method.  

However, the above method is inefficient. In fact, for large numbers that are often used in public-key cryptography, factoring is not even computationally feasible. We will see a much more efficient method, called the \emph{Euclidean Algorithm}.



\vspace{5mm}

\noindent\textbf{Definition 14:} (Relatively Prime)
\label{def:relprime}
\\Let $a,b \in \mathbb{Z}$. We say $a$ and $b$ are relatively prime if $\gcd(a,b) = 1$.

\vspace{5mm}

\noindent\textbf{Example 15:}
\label{ex:relprime}
\\From \hyperref[ex:primefact]{Example 13}, 20 and 621 are relatively prime.

\vspace{15mm}

\subsection{Euclidean Algorithm}

\noindent\textbf{Lemma 16:}
\label{lem:euclid}
\\Let $a=bq+r$, where $a,b,q,r \in \mathbb{Z}$. Then $\gcd(a,b) = \gcd(b,r)$.
This forms the basis for the Euclidean Algorithm:

\begin{enumerate}
    \item Let $c = a \bmod b$.
    \item If $c = 0$, then $\gcd(a,b) = b$. Stop.
    \item Otherwise, the answer is $\gcd(b,c)$.
\end{enumerate}


\noindent\textbf{Example 17:}
\label{ex:euclid360}
\begin{align*}
360 \bmod 84 &= 24 \\
84 \bmod 24 &= 12 \\
24 \bmod 12 &= 0
\end{align*}
so, $\gcd(360,84) = 12$.

\vspace{5mm}

\noindent\textbf{Example 18:}
\label{ex:euclid720}
\\to find $\gcd(720,26)$ using the Euclidean Algorithm:
\begin{align*}
720 \bmod 26 &= 16 \\   
26 \bmod 16 &= 10 \\
16 \bmod 10 &= 6 \\
10 \bmod 6 &= 4 \\
6 \bmod 4 &= 2 \\
4 \bmod 2 &= 0
\end{align*}
so, $\gcd(720,26) = 2$.


\noindent\textbf{Theorem 19:}
\label{thm:bezout}
\\Let $a,b \in \mathbb{Z}$, at least one nonzero. 
The gcd $d$ of $a$ and $b$ can be written as:
\[
d = ax + by
\]
for some integers $x$ and $y$.  
\\We can use the Euclidean Algorithm to find $x$ and $y$.  
\\Recall from \hyperref[thm:division]{Theorem 2} that for any integers 
$a$ and $b$ with $b>0$, if we divide $a$ by $b$ we obtain $r=a \bmod b$ such that:
(the remainder) and $q=a \div b$ (the quotient), and we can write $a=bq+r$.
\\Note that in the first step of the Euclidean Algorithm, we only kept track of the 
remainder ($a \bmod b$) but now we will also keep track of the quotient ($a \div b$),
and will write each line in the form $a=bq+r$.

\newpage
\section{Recurrence Relations}
\subsection*{Coming soon...}

\newpage
\section{Graphs and Trees}
\subsection*{Coming soon...}

\newpage
\section{Practice problems}
\subsection*{week 1}
\addcontentsline{toc}{subsection}{week 1}


\subsubsection*{Section 4.1 (Divisibility and Modular Arithmetic)}
\addcontentsline{toc}{subsubsection}{Section 4.1 (Divisibility and Modular Arithmetic)}

Questions: 3, 5, 13, 21, 31
\\\\
\underline{3.prove that if $a \mid b$ then $a \mid bc$ for all integers $c$:}
\\\indent by the definition of Divisibility, we know that if $a \mid b$, then $b = ak$ for some integer $k$.
 if we multiply both sides by $c$, we get $bc = akc$.
 since $akc = a(kc)$, and $kc$ is an integer, we can say that $a \mid bc$.   
\\\\
\underline{5.Show that if $a \mid b$ and $b\mid a$ where $a,b \in \mathbb{Z}$, then $a=b$ or $a=-b$:}
\\\indent by the definition of Divisibility, we know that if $a \mid b$, then $b = ak$ for some integer $k$.
 similarly, if $b \mid a$, then $a = bm$ for some integer $m$.
 \\substituting the first equation into the second, we get $a = (ak)m$, or $a = akm$.
 dividing both sides by $a$ (which is nonzero), we get $1 = km$.
 since $k$ and $m$ are integers, the only way for their product to be 1 is if both are 1 or both are -1.
 thus, if $k=1$, then $b=a$, and if $k=-1$, then $b=-a$.
\\\\
\underline{13.What are the quotient and remainder when:}
\begin{enumerate}[label=(\alph*)]
    \item 19 is divided by 7
    \[
    19 = 2 \cdot 7 + 5 \quad \Rightarrow \quad 19 \div 7 = 2, \quad 19 \bmod 7 = 5
    \]
    \item -111 is divided by 11
    \[
    -111 = -11 \cdot 11 + 10 \quad \Rightarrow \quad -111 \div 11 = -11, \quad -111 \bmod 11 = 10
    \]
    \item 789 is divided by 23
    \[
    789 = 34 \cdot 23 + 7 \quad \Rightarrow \quad 789 \div 23 = 34, \quad 789 \bmod 23 = 7
    \]
    \item 1001 is divided by 13
    \[
    1001 = 77 \cdot 13 + 0 \quad \Rightarrow \quad 1001 \div 13 = 77, \quad 1001 \bmod 13 = 0
    \]
    \item 0 is divided by 19
    \[
    0 = 0 \cdot 19 + 0 \quad \Rightarrow \quad 0 \div 19 = 0, \quad 0 \bmod 19 = 0
    \]
    \item 3 is divided by 5
    \[
    3 = 0 \cdot 5 + 3 \quad \Rightarrow \quad 3 \div 5 = 0, \quad 3 \bmod 5 = 3
    \]
    \item -1 is divided by 3
    \[
    -1 = -1 \cdot 3 + 2 \quad \Rightarrow \quad -1 \div 3 = -1, \quad -1 \bmod 3 = 2
    \]
    \item 4 is divided by 1
    \[
    4 = 4 \cdot 1 + 0 \quad \Rightarrow \quad 4 \div 1 = 4, \quad 4 \bmod 1 = 0
    \]
\end{enumerate}

\vspace{5mm}
\noindent\underline{21.Let $m$ be a positive integer. Show that $a \equiv b \pmod{m}$ if $a \bmod m = b \bmod m$:}
\\\indent by the definition of congruence modulo $m$, we know that if $a \equiv b \pmod{m}$, then $m \mid (a-b)$.
 this means that $a-b = mk$ for some integer $k$.
 adding $b$ to both sides, we get $a = mk + b$.
 taking both sides modulo $m$, we get $a \bmod m = (mk + b) \bmod m$.
 since $mk \bmod m = 0$, we have $a \bmod m = b \bmod m$.

\vspace{5mm}

\noindent\underline{31.Find the integer $a$ such that:}
\begin{enumerate}[label=(\alph*)]
    \item $a \equiv -15 \pmod{27}$ and $-26 \leq a \leq 0$
\[
a = -15 \quad \text{(already in the interval, so the solution is $a=-15$).}
\]

\item $a \equiv 24 \pmod{31}$ and $-15 \leq a \leq 15$
\[
a = 24 - 31 = -7 \quad \text{(in the interval, so the solution is $a=-7$).}
\]

\item $a \equiv 99 \pmod{41}$ and $100 \leq a \leq 140$
\[
a = 99 + 41 = 140 \quad \text{(in the interval, so the solution is $a=140$).}
\]

\end{enumerate}



\newpage
\subsubsection*{Section 4.3 (Primes and Greatest Common Divisors)}
\addcontentsline{toc}{subsubsection}{Section 4.3 (Primes and Greatest Common Divisors)}
Questions: 3, 13, 17, 19, 25, 30, 31, 33, 39, 41, 43, 45
\\\\

\noindent\underline{3.Find the prime factorization of each of the following integers:}
\begin{enumerate}[label=(\alph*)]
    \item 88
    \[
    88 = 2 \cdot 2 \cdot 2 \cdot 11 = 2^3 \cdot 11
    \]
    \item 126
    \[
    126 = 2 \cdot 3 \cdot 3 \cdot 7 = 2^1 \cdot 3^2 \cdot 7
    \]
    \item 729
    \[
    729 = 3 \cdot 3 \cdot 3 \cdot 3 \cdot 3 \cdot 3 = 3^6
    \]
    \item 1001
    \[
    1001 = 7 \cdot 11 \cdot 13 = 7^1 \cdot 11^1 \cdot 13^1
    \]
    \item 1111
    \[
    1111 = 11 \cdot 101 = 11^1 \cdot 101^1
    \]
    \item 909090
    \[
    909090 = 2 \cdot 3 \cdot 3 \cdot 3 \cdot 5 \cdot 7  \cdot 13 \cdot 37 = 2^1 \cdot 3^3 \cdot 5^1 \cdot 7^1  \cdot 13^1 \cdot 37^1
    \]
\end{enumerate}

\vspace{3.3mm}
\noindent\underline{13.} prove or disprove that there are three consecutive odd positive integers that are primes, that is odd primes of  the form p, p+2, and p+4:
\rule{\linewidth}{0.5pt}
\indent Let $p$ be an odd prime. then consider $p, p+2, p+4$ in $\pmod{3}$.
\begin{itemize}
    \item If $p \equiv 0 \pmod{3}$, then $p=3$ (since $p$ is prime). Then $p+2=5$ (prime), but $p+4=7$ (also prime). So this case works.
    \item If $p \equiv 1 \pmod{3}$, then $p+2 \equiv 0 \pmod{3}$. Since $p+2 > 3$, it is divisible by 3 and not prime.
    \item If $p \equiv 2 \pmod{3}$, then $p+4 \equiv 0 \pmod{3}$. Since $p+4 > 3$, it is divisible by 3 and not prime.
\end{itemize}
Therefore the only set of three consecutive odd positive integers that are primes is 3, 5, and 7.

\newpage
\noindent\underline{17. Determine whether the integers in each of these sets are pairwise relatively prime:}
\begin{enumerate}[label=(\alph*)]
    \item $11, 15 ,19$
    \[
    \gcd(11, 15) = 1, \quad \gcd(11, 19) = 1, \quad \gcd(15, 19) = 1
    \]
    they are relatively prime.
    \item $14, 15 ,21$
    \[
    \gcd(14, 15) = 1, \quad \gcd(14, 21) = 7, \quad \gcd(15, 21) = 3
    \]
    they are not relatively prime.
    \item $12, 17, 31 ,37$
    \[
    \gcd(12, 17) = 1, \quad \gcd(12, 31) = 1, \quad \gcd(12, 37) = 1, \] 
    \[\gcd(17, 31) = 1, \quad \gcd(17, 37) = 1, \quad \gcd(31, 37) = 1\]
    they are relatively prime.
    \item $7, 8, 9, 11$
    \[
    \gcd(7, 8) = 1, \quad \gcd(7, 9) = 1, \quad \gcd(7, 11) = 1, \] 
    \[\gcd(8, 9) = 1, \quad \gcd(8, 11) = 1, \quad \gcd(9, 11) = 1\]
    they are relatively prime.
\end{enumerate}

\vspace{5mm}
\noindent\underline{19. Show that if $2^n-1$ is prime, then $n$ is prime:}
\\\indent By contrapositive, if $n$ is not prime, then $2^n-1$ is not prime. 
Let $n = ab$, where $a,b > 1$. Then $2^n -1 = 2^{ab} -1$.  This can be factored as: 
$(2^a-1)(2^{a(b-1)} + 2^{a(b-2)} + ... + 2^a + 1)$.
Since both factors are greater than 1, $2^n - 1$ is not prime.

\vspace{5mm}
\noindent\underline{25. What are the greatest common divisors of these pairs of integers:}
\begin{enumerate}[label=(\alph*)]
    \item $3^7 \cdot 5^3 \cdot 7^3$ and $2^{11} \cdot 3^5 \cdot 5^9$
\[
\gcd(3^7 \cdot 5^3 \cdot 7^3, 2^{11} \cdot 3^5 \cdot 5^9) = 3^{\min(7,5)} \cdot 5^{\min(3,9)} \cdot 7^{\min(3,0)} = 3^5 \cdot 5^3 \cdot 7^0 = 3^5 \cdot 5^3
\]
    \item $11 \cdot 13 \cdot 17$ and $2^9 \cdot 3^7 \cdot 5^5 \cdot 7^3$
    \item $23^{31}$ and $23^{17}$
    \item $41 \cdot 43 \cdot 53$ and $41 \cdot 43 \cdot 53$
    \item $3^{13} \cdot 5^{17}$ and $2^{12} \cdot 7^{21}$
    \item $1111$ and $0$

    
\end{enumerate}

\end{document}
