\documentclass[12pt]{article}
\usepackage[utf8]{inputenc}
\usepackage[colorlinks=true,linkcolor=blue]{hyperref}
\usepackage{amsmath, amssymb}
\usepackage{enumitem}
\usepackage{geometry}
\geometry{margin=1in}

\DeclareMathOperator{\lcm}{lcm}

\title{CISC203 notes\\Discrete Structures II}
\author{Andrew Aquino}
\date{}

\begin{document}

\maketitle

\section*{Introduction}

To whom this may concern, these are some notes for CISC203, 
a course I took in the fall of 2025. 
I am writing these notes for my own benefit, 
and I make no claims about their accuracy or completeness. 
Use at your own risk.

ps on 09/16/2025, I was intoxicated while writing these notes :3

Check me out at \url{https://github.com/andrewSmellz}

\tableofcontents
\newpage

\section{Group Theory}

\subsection{Division}

\noindent\textbf{Definition 1:} ($a \mid b$) 
\label{def:divides}
\\Let $a, b \in \mathbb{Z}$, with $a \neq 0$.  
If $b = ak$ for some $k \in \mathbb{Z}$, then we say that $a$ divides $b$,  
or that $a$ is a divisor of $b$. 
\[\text{This is denoted as: } a \mid b\]

\vspace{5mm}

\noindent\textbf{Theorem 2:} (Division Algorithm) 
\label{thm:division}
\\Let $a, b \in \mathbb{Z}$, with $b > 0$.  
There exists a unique pair of integers $q$ and $r$ such that:

\[
a = qb + r \quad \text{where} \quad 0 \leq r < b
\]

\noindent This expresses $a$ as a multiple of $b$ plus a remainder $r$.
Additionally, we call $b$ the divisor, $a$ the dividend, $q$ the quotient, and $r$ the remainder.

\vspace{5mm}

\noindent\textbf{Definition 3:} (Division and Modulus)
\label{def:divmod}
\\Let $a, b \in \mathbb{Z}$, with $b > 0$. 
Then we define the division and modulus operations as follows:

\[
q = a \div b, \quad r = a \bmod b
\]

where $q$ and $r$ are the unique pair of integers (by Theorem \hyperref[thm:division]{2}) for which $a = qb + r$ and $0 \leq r < b$.

\vspace{5mm}

\noindent\textbf{Definition 4:} (Congruence Modulo $n$)
\label{def:congruence}
\\Let $x,y,n \in \mathbb{Z}$, with $n > 0$. If $n \mid (x - y)$, we can say that $x$ and $y$ are congruent modulo $n$. 
This is denoted as: $x \equiv y \pmod{n}$.
The set of all integers congruent to an integer $a$ modulo $n$ is called the congruence class of $a$ modulo $n$.

\vspace{5mm}

\noindent\textbf{Example 5:} 
\label{ex:modexample}
\\ $53 \equiv 23 \pmod{10}$ means that $53-23=30$ is a multiple of 10. 
\\However, $53 \bmod 10 = 3$ and $23 \bmod 10 = 3$, meaning that the remainder of $53 \div 10$ is $3$.

\vspace{5mm}

\noindent\textbf{Theorem 6:} 
\label{thm:modiff}
\\Let $a,b,n \in \mathbb{Z}$, with $n > 0$. 
Then 
\[
a \equiv b \pmod{n} \iff a \bmod n = b \bmod n.
\]

\vspace{5mm}

\noindent\textbf{Example 7:}
\label{ex:modexample2}
\[
9 \equiv 17 \pmod{4} \text{ is true, so we also have } 
9\bmod 4 = 1 \text{ and } 17 \bmod 4 = 1.
\]

\vspace{15mm}

\subsection{Greatest Common Divisor}

\noindent\textbf{Definition 8:} (Common Divisor)
\label{def:commondivisor}
\\Let $a,b \in \mathbb{Z}$. If an integer $d$ divides both $a$ and $b$, we say that $d$ is a \textbf{common divisor} of $a$ and $b$.

\vspace{5mm}

\noindent\textbf{Example 9:}
\label{ex:commondvs}
\\The common divisors of 12 and 18 are $\pm1, \pm2, \pm3, \text{ and } \pm6$.
\\The common divisors of 25 and 50 are $\pm1, \pm5, \pm25$.

\vspace{5mm}

\noindent\textbf{Definition 10:} (Greatest Common Divisor)
\label{def:gcd}
\\Let $a,b \in \mathbb{Z}$. An integer $d\ge 0$ is the greatest common divisor of $a$ and $b$ if:
\begin{enumerate}
    \item $d$ divides $a$ and $d$ divides $b$ (so $d$ is a common divisor), and
    \item for every common divisor $e$ of $a$ and $b$ we have $e \mid d$.
\end{enumerate}
We denote the greatest common divisor of $a$ and $b$ by $\gcd(a,b)$.

\vspace{5mm}

\noindent\textbf{Example 11:}
\label{ex:gcdnaive}
\\The greatest common divisor of 18 and 12 is 6. Using the naive method: 
\begin{enumerate}
    \item Find all divisors of $a$.
    \item Find all divisors of $b$.
    \item Choose the largest positive number that is a divisor of both $a$ and $b$.
\end{enumerate}

\vspace{5mm}

\noindent\textbf{Theorem 12:} (Fundamental Theorem of Arithmetic)
\label{thm:fta}
\\Every integer greater than 1 can be written uniquely (up to ordering of equal primes) as a product of primes.

\vspace{5mm}

\noindent\textbf{Example 13:}
\label{ex:primefact}
\\The prime factorizations of $20, 23, 288$, and $621$ are:
\[
r_0 = 20 = 2 \cdot 2 \cdot 5 = 2^2 \cdot 5,
\]
\[
r_1 = 23 = 23,
\]
\[
r_2 = 288 = 2 \cdot 2 \cdot 2 \cdot 2 \cdot 2 \cdot 3 \cdot 3 = 2^5 \cdot 3^2,
\]
\[
r_3 = 621 = 3 \cdot 3 \cdot 3 \cdot 23 = 3^3 \cdot 23.
\]
We can find the greatest common divisor of two numbers by taking the product of all common prime factors with the minimum exponents. That is, 
\[
\gcd(r_0, r_2) = 2^2 = 4 \quad \text{and} \quad \gcd(r_2, r_3) = 3^2 = 9.
\]
Since \(r_0\) does not have any common prime factors with \(r_3\), we have
\[
\gcd(r_0, r_3) = 1.
\]

We can also find the least common multiple of two numbers using the maximum exponents in their prime factorizations.  

However, the above method is inefficient for very large numbers. We will see a much more efficient method, called the \emph{Euclidean Algorithm}.

\vspace{5mm}

\noindent\textbf{Definition 14:} (Relatively Prime)
\label{def:relprime}
\\Let $a,b \in \mathbb{Z}$. We say $a$ and $b$ are relatively prime if $\gcd(a,b) = 1$.

\vspace{5mm}

\noindent\textbf{Example 15:}
\label{ex:relprime}
\\From \hyperref[ex:primefact]{Example 13}, 20 and 621 are relatively prime.

\vspace{15mm}

\subsection{Euclidean Algorithm}

\noindent\textbf{Lemma 16:}
\label{lem:euclid}
\\Let $a=bq+r$, where $a,b,q,r \in \mathbb{Z}$. Then $\gcd(a,b) = \gcd(b,r)$.
This forms the basis for the Euclidean Algorithm:

\begin{enumerate}
    \item Let $c = a \bmod b$.
    \item If $c = 0$, then $\gcd(a,b) = b$. Stop.
    \item Otherwise, the answer is $\gcd(b,c)$.
\end{enumerate}

\noindent\textbf{Example 17:}
\label{ex:euclid360}
\begin{align*}
360 \bmod 84 &= 24, \\
84 \bmod 24 &= 12, \\
24 \bmod 12 &= 0.
\end{align*}
so, $\gcd(360,84) = 12$.

\vspace{5mm}

\noindent\textbf{Example 18:}
\label{ex:euclid720}
\\To find $\gcd(720,26)$ using the Euclidean Algorithm:
\begin{align*}
720 \bmod 26 &= 18 \quad(26\cdot 27 = 702,\ 720-702=18), \\   
26 \bmod 18 &= 8, \\
18 \bmod 8 &= 2, \\
8 \bmod 2 &= 0.
\end{align*}
so, $\gcd(720,26) = 2$.

\newpage
\noindent\textbf{Theorem 19:}
\label{thm:bezout}
\\Let $a,b \in \mathbb{Z}$, not both zero. 
The gcd $d$ of $a$ and $b$ can be written as:
\[
d = ax + by
\]
for some integers $x$ and $y$.  
\\We can use the Euclidean Algorithm to find $x$ and $y$.  
\\Recall from \hyperref[thm:division]{Theorem 2} that for any integers 
$a$ and $b$ with $b>0$, if we divide $a$ by $b$ we obtain $r=a \bmod b$ (the remainder) and $q=a \div b$ (the quotient), and we can write $a=bq+r$.
\\Note that in the first step of the Euclidean Algorithm, we only kept track of the 
remainder ($a \bmod b$) but now we will also keep track of the quotient ($a \div b$),
and will write each line in the form $a=bq+r$.

\newpage
\section{Recurrence Relations}
\subsection*{Coming soon...}

\newpage
\section{Graphs and Trees}
\subsection*{Coming soon...}

\newpage
\section{Practice problems}
\subsection*{week 1}
\addcontentsline{toc}{subsection}{week 1}


\subsubsection*{Section 4.1 (Divisibility and Modular Arithmetic)}
\addcontentsline{toc}{subsubsection}{Section 4.1 (Divisibility and Modular Arithmetic)}

Questions: 3, 5, 13, 21, 31
\\\\
\underline{3.\;Prove that if $a \mid b$ then $a \mid bc$ for all integers $c$:}
\\\indent By the definition of divisibility, if $a \mid b$ then $b = ak$ for some integer $k$.
Multiplying both sides by $c$ gives $bc = akc$.
Since $akc = a(kc)$ and $kc$ is an integer, we have $a \mid bc$.   

\vspace{5mm}
\underline{5.\;Show that if $a \mid b$ and $b\mid a$ where $a,b \in \mathbb{Z}$, then $a=b$ or $a=-b$:}
\\\indent By the definition of divisibility, if $a \mid b$ then $b = ak$ for some integer $k$.
Similarly, if $b \mid a$, then $a = bm$ for some integer $m$.
Substituting the first equation into the second gives $a = (ak)m$, or $a = akm$.
If $a\neq 0$, dividing both sides by $a$ yields $1 = km$.
Since $k$ and $m$ are integers, the only integer solutions to $km=1$ are $k=m=1$ or $k=m=-1$.
Thus, if $k=1$, then $b=a$, and if $k=-1$, then $b=-a$.
(If $a=0$, then the divisibility statements imply $b=0$, and the conclusion $a=b$ also holds.)

\vspace{5mm}
\underline{13.\;What are the quotient and remainder when:}
\begin{enumerate}[label=(\alph*)]
    \item 19 is divided by 7
    \[
    19 = 2 \cdot 7 + 5 \quad \Rightarrow \quad 19 \div 7 = 2, \quad 19 \bmod 7 = 5
    \]
    \item -111 is divided by 11
    \[
    -111 = -11 \cdot 11 + 10 \quad \Rightarrow \quad -111 \div 11 = -11, \quad -111 \bmod 11 = 10
    \]
    (Note: this follows the convention that remainders are in $0,\dots,10$.)
    \item 789 is divided by 23
    \[
    789 = 34 \cdot 23 + 7 \quad \Rightarrow \quad 789 \div 23 = 34, \quad 789 \bmod 23 = 7
    \]
    \item 1001 is divided by 13
    \[
    1001 = 77 \cdot 13 + 0 \quad \Rightarrow \quad 1001 \div 13 = 77, \quad 1001 \bmod 13 = 0
    \]
    \item 0 is divided by 19
    \[
    0 = 0 \cdot 19 + 0 \quad \Rightarrow \quad 0 \div 19 = 0, \quad 0 \bmod 19 = 0
    \]
    \item 3 is divided by 5
    \[
    3 = 0 \cdot 5 + 3 \quad \Rightarrow \quad 3 \div 5 = 0, \quad 3 \bmod 5 = 3
    \]
    \item -1 is divided by 3
    \[
    -1 = -1 \cdot 3 + 2 \quad \Rightarrow \quad -1 \div 3 = -1, \quad -1 \bmod 3 = 2
    \]
    \item 4 is divided by 1
    \[
    4 = 4 \cdot 1 + 0 \quad \Rightarrow \quad 4 \div 1 = 4, \quad 4 \bmod 1 = 0
    \]
\end{enumerate}

\vspace{5mm}
\noindent\underline{21.\;Let $m$ be a positive integer. Show that $a \equiv b \pmod{m}$ if $a \bmod m = b \bmod m$:}
\\\indent Suppose $a \bmod m = b \bmod m$. Then there exist integers $q_a,q_b$ and a remainder $r$ with $0\le r<m$ such that
\[
a = m q_a + r,\qquad b = m q_b + r.
\]
Subtracting gives $a-b = m(q_a-q_b)$, so $m\mid (a-b)$. Hence $a\equiv b\pmod m$.

\vspace{5mm}

\noindent\underline{31.\;Find the integer $a$ such that:}
\begin{enumerate}[label=(\alph*)]
    \item $a \equiv -15 \pmod{27}$ and $-26 \leq a \leq 0$
\[
a = -15 \quad \text{(already in the interval, so the solution is $a=-15$).}
\]

\item $a \equiv 24 \pmod{31}$ and $-15 \leq a \leq 15$
\[
a = 24 - 31 = -7 \quad \text{(in the interval, so the solution is $a=-7$).}
\]

\item $a \equiv 99 \pmod{41}$ and $100 \leq a \leq 140$
\[
a = 99 + 41 = 140 \quad \text{(in the interval, so the solution is $a=140$).}
\]

\end{enumerate}



\newpage
\subsubsection*{Section 4.3 (Primes and Greatest Common Divisors)}
\addcontentsline{toc}{subsubsection}{Section 4.3 (Primes and Greatest Common Divisors)}
Questions: 3, 13, 17, 19, 25, 30, 31, 33, 39, 41, 43, 45
\\\\

\noindent\underline{3.\;Find the prime factorization of each of the following integers:}
\begin{enumerate}[label=(\alph*)]
    \item 88
    \[
    88 = 2 \cdot 2 \cdot 2 \cdot 11 = 2^3 \cdot 11
    \]
    \item 126
    \[
    126 = 2 \cdot 3 \cdot 3 \cdot 7 = 2 \cdot 3^2 \cdot 7
    \]
    \item 729
    \[
    729 = 3 \cdot 3 \cdot 3 \cdot 3 \cdot 3 \cdot 3 = 3^6
    \]
    \item 1001
    \[
    1001 = 7 \cdot 11 \cdot 13
    \]
    \item 1111
    \[
    1111 = 11 \cdot 101
    \]
    \item 909090
    \[
    909090 = 2 \cdot 3^3 \cdot 5 \cdot 7 \cdot 13 \cdot 37 = 2 \cdot 3^3 \cdot 5 \cdot 7 \cdot 13 \cdot 37
    \]
\end{enumerate}

\vspace{3.3mm}
\noindent\underline{13.} prove or disprove that there are three consecutive odd positive integers that are primes, that is odd primes of the form $p, p+2, p+4$:
\\\rule{\linewidth}{0.5pt}
\indent Let $p$ be an odd prime, and consider $p, p+2, p+4$ modulo $3$.
\begin{itemize}
    \item If $p \equiv 0 \pmod{3}$, then $p=3$ (since $p$ is prime). Then $p+2=5$ (prime), and $p+4=7$ (also prime). So this case gives the triple $3,5,7$.
    \item If $p \equiv 1 \pmod{3}$, then $p+2 \equiv 0 \pmod{3}$. Since $p+2 > 3$, it is divisible by 3 and not prime.
    \item If $p \equiv 2 \pmod{3}$, then $p+4 \equiv 0 \pmod{3}$. Since $p+4 > 3$, it is divisible by 3 and not prime.
\end{itemize}
Therefore the only set of three consecutive odd positive integers that are all prime is $3,5,7$.

\newpage
\noindent\underline{17.\;Determine whether the integers in each of these sets are pairwise relatively prime:}
\begin{enumerate}[label=(\alph*)]
    \item $11, 15 ,19$
    \[
    \gcd(11, 15) = 1, \quad \gcd(11, 19) = 1, \quad \gcd(15, 19) = 1
    \]
    they are pairwise relatively prime.
    \item $14, 15 ,21$
    \[
    \gcd(14, 15) = 1, \quad \gcd(14, 21) = 7, \quad \gcd(15, 21) = 3
    \]
    they are not pairwise relatively prime.
    \item $12, 17, 31 ,37$
    \[
    \gcd(12, 17) = 1, \quad \gcd(12, 31) = 1, \quad \gcd(12, 37) = 1, \] 
    \[\gcd(17, 31) = 1, \quad \gcd(17, 37) = 1, \quad \gcd(31, 37) = 1
    \]
    they are pairwise relatively prime.
    \item $7, 8, 9, 11$
    \[
    \gcd(7, 8) = 1, \quad \gcd(7, 9) = 1, \quad \gcd(7, 11) = 1, \] 
    \[\gcd(8, 9) = 1, \quad \gcd(8, 11) = 1, \quad \gcd(9, 11) = 1
    \]
    they are pairwise relatively prime.
\end{enumerate}

\vspace{5mm}
\noindent\underline{19.\;Show that if $2^n-1$ is prime, then $n$ is prime:}
\\\indent By contrapositive, if $n$ is not prime (so $n = ab$ with $a,b>1$), then
\[
2^n -1 = 2^{ab} -1 = (2^a-1)\left(2^{a(b-1)} + 2^{a(b-2)} + \cdots + 2^a + 1\right),
\]
so \(2^n-1\) is composite. Therefore if \(2^n-1\) is prime then \(n\) must be prime.

\vspace{5mm}
\noindent\underline{25.\;What are the greatest common divisors of these pairs of integers:}
\begin{enumerate}[label=(\alph*)]
    \item $3^7 \cdot 5^3 \cdot 7^3$ and $2^{11} \cdot 3^5 \cdot 5^9$
\[
\gcd(3^7 \cdot 5^3 \cdot 7^3, 2^{11} \cdot 3^5 \cdot 5^9) = 3^{\min(7,5)} \cdot 5^{\min(3,9)} \cdot 7^{\min(3,0)} = 3^5 \cdot 5^3
\]
    \item $11 \cdot 13 \cdot 17$ and $2^9 \cdot 3^7 \cdot 5^5 \cdot 7^3$
\[
\gcd(11 \cdot 13 \cdot 17, 2^9 \cdot 3^7 \cdot 5^5 \cdot 7^3) = 1
\]
    \item $23^{31}$ and $23^{17}$
\[
\gcd(23^{31}, 23^{17}) = 23^{17}
\]
    \item $41 \cdot 43 \cdot 53$ and $41 \cdot 43 \cdot 53$
\[
\gcd(41 \cdot 43 \cdot 53, 41 \cdot 43 \cdot 53) = 41 \cdot 43 \cdot 53
\]
    \item $3^{13} \cdot 5^{17}$ and $2^{12} \cdot 7^{21}$
\[
\gcd(3^{13} \cdot 5^{17}, 2^{12} \cdot 7^{21}) = 1
\]
    \item $1111$ and $0$
\[
\gcd(1111, 0) = 1111.
\]
\end{enumerate}

\vspace{15mm}
\textbf{From this point on, I have begun sipping a cut water, moose on the line, money on the mind.}


\noindent\underline{30.}If the product of two integers is $2^7\cdot 3^8 \cdot 5^2 \cdot 7^{11}$ and their greatest common divisor is $2^3 \cdot 3^4 \cdot 5$, what is their least common multiple?
 \\ \rule{\linewidth}{0.5pt}
By definition, for positive integers $a$ and $b$,
\[
a \cdot b = \gcd(a,b) \cdot \lcm(a,b).
\]
Therefore
\[
\lcm(a,b) = \frac{a \cdot b}{\gcd(a,b)} = \frac{2^7\cdot 3^8 \cdot 5^2 \cdot 7^{11}}{2^3 \cdot 3^4 \cdot 5} = 2^{4} \cdot 3^{4} \cdot 5^{1} \cdot 7^{11}.
\]

\vspace{5mm}

\noindent\underline{31.}Show that if $a$ and $b$ are positive integers, then $ab=\gcd(a,b) \cdot \lcm(a,b)$:
\\ \rule{\linewidth}{0.5pt}
\\\indent By the prime factorization theorem, we can write $a$ and $b$ as:
\[a = p_1^{e_1} \cdot p_2^{e_2} \cdot \dots \cdot p_k^{e_k}\]
\[b = p_1^{f_1} \cdot p_2^{f_2} \cdot \dots \cdot p_k^{f_k}\]
where $p_1, p_2, \dots, p_k$ are the primes appearing in either factorization (exponent possibly zero for some primes).
The gcd and lcm can be written as:
\[\gcd(a,b) = p_1^{\min(e_1,f_1)} \cdot p_2^{\min(e_2,f_2)} \cdot \dots \cdot p_k^{\min(e_k,f_k)}\]
\[\lcm(a,b) = p_1^{\max(e_1,f_1)} \cdot p_2^{\max(e_2,f_2)} \cdot \dots \cdot p_k^{\max(e_k,f_k)}\]
Multiplying these together, we get:
\[
 \gcd(a,b) \cdot \lcm(a,b) = p_1^{\min(e_1,f_1) + \max(e_1,f_1)} \cdot \dots \cdot p_k^{\min(e_k,f_k) + \max(e_k,f_k)}.
\]
Since $\min(x,y) + \max(x,y) = x + y$, we have:
\[
 \gcd(a,b) \cdot \lcm(a,b) = p_1^{e_1 + f_1} \cdot \dots \cdot p_k^{e_k + f_k} = a \cdot b.
\]

\vspace{5mm}
\noindent\underline{33.\;Use the Euclidean algorithm to find:}
\begin{enumerate}[label=(\alph*)]
    \item $\gcd(12,18)$
    \[
    18 \bmod 12 = 6,\quad 12 \bmod 6 = 0
    \]
    so, $\gcd(12,18) = 6$.
    \item $\gcd(111,201)$
    \[
    201 \bmod 111 = 90,\quad 111 \bmod 90 = 21,\quad 90 \bmod 21 = 6,\quad 21 \bmod 6 = 3,\quad 6 \bmod 3 = 0
    \]
    so, $\gcd(111,201) = 3$.
    \item $\gcd(1001,1331)$
    \[
    1331 \bmod 1001 = 330,\quad 1001 \bmod 330 = 11,\quad 330 \bmod 11 = 0
    \]
    so, $\gcd(1001,1331) = 11$.
    \item $\gcd(12345,54321)$
    \[
    54321 \bmod 12345 = 4941,\quad 12345 \bmod 4941 = 2463,\quad 4941 \bmod 2463 = 15,
    \]
    \[
    2463 \bmod 15 = 3,\quad 15 \bmod 3 = 0
    \]
    so, $\gcd(12345,54321) = 3$.
    \item $\gcd(1000,5040)$
    \[
    5040 \bmod 1000 = 40,\quad 1000 \bmod 40 = 0
    \]
    so, $\gcd(1000,5040) = 40$.
    \item $\gcd(9888,6060)$
    \[
    9888 \bmod 6060 = 3828,\quad 6060 \bmod 3828 = 2232,\quad 3828 \bmod 2232 = 1596,
    \]
    \[
    2232 \bmod 1596 = 636,\quad 1596 \bmod 636 = 324,\quad 636 \bmod 324 = 312,
    \]
    \[
    324 \bmod 312 = 12,\quad 312 \bmod 12 = 0
    \]
    so, $\gcd(9888,6060) = 12$.
\end{enumerate}

\newpage
\noindent\underline{39.}Express the greatest common divisor of each of these pairs of integers as a linear combination of the integers:
\\\rule{\linewidth}{0.5pt}
\begin{enumerate} [label=(\alph*)]
    \item 10,11 
    \[
    \begin{aligned}
    11 &= 1 \cdot 10 + 1, \\
    10 &= 10 \cdot 1 + 0.
    \end{aligned}
    \]
Thus, $\gcd(10,11) = 1$.

Back-substitute:
    \[
    1 = 11 - 1\cdot 10,
    \]
so
\[
1 = 10(-1) + 11(1).
\]

    \item 21,44

\[44 = 2\cdot21 + 2,\quad 21 = 10\cdot 2 + 1,\quad 2 = 2\cdot 1 + 0.\]

Thus, $\gcd(21,44) = 1$.

Back-substitution:
\[
1 = 21 - 10\cdot 2 = 21 - 10(44 - 2\cdot21) = 21\cdot21 - 44\cdot10,
\]
so
\[
1 = 21(21) + 44(-10).
\]

\item 36,48

\[
48 = 1\cdot36 + 12,\quad 36 = 3\cdot12 + 0.
\]

Thus, $\gcd(36,48) = 12$.

Back-substitution:
\[
12 = 48 - 1\cdot36,
\]
so
\[
12 = 36(-1) + 48(1).
\]

\item 34,55

\[
\begin{aligned}
55 &= 1\cdot34 + 21, \\
34 &= 1\cdot21 + 13, \\
21 &= 1\cdot13 + 8, \\
13 &= 1\cdot8 + 5, \\
8 &= 1\cdot5 + 3, \\
5 &= 1\cdot3 + 2, \\
3 &= 1\cdot2 + 1, \\
2 &= 2\cdot1 + 0.
\end{aligned}
\]

Back-substitution gives the linear combination
\[
1 = 34(-21) + 55(13).
\]

\item 117,213

\[
\begin{aligned}
213 &= 1\cdot117 + 96, \\
117 &= 1\cdot96 + 21, \\
96 &= 4\cdot21 + 12, \\
21 &= 1\cdot12 + 9, \\
12 &= 1\cdot9 + 3, \\
9 &= 3\cdot3 + 0.
\end{aligned}
\]

Back-substitution yields
\[
3 = 117(-20) + 213(11).
\]

    \item 0,223

If one argument is 0, say $\gcd(0,223)=223$ and we can write
\[
223 = 0\cdot 0 + 223\cdot 1,
\]
so $\gcd(0,223) = 223 = 0(0) + 223(1)$.

    \item 123,2347

(See the extended Euclidean algorithm in Section 45 for a full worked example; result below.)

    \item 3454,4666

(Left as an exercise; use the Euclidean algorithm and back-substitute.)

    \item 9999,11111

(Left as an exercise; use the Euclidean algorithm and back-substitute.)
\end{enumerate}

\vspace{5mm}
\noindent\underline{41.}Use the extended Euclidean algorithm to express $\gcd(26,91)$ as a linear combination of 26 and 91:
\\ \rule{\linewidth}{0.5pt}
\[
91 = 3 \cdot 26 + 13,\quad 26 = 2 \cdot 13 + 0.
\]

Thus, $\gcd(26,91) = 13$.

Back-substitution:
\[
13 = 91 - 3\cdot 26,
\]
so
\[
13 = 26(-3) + 91(1).
\]

\vspace{5mm}
\noindent\underline{43.}Use the extended Euclidean algorithm to express $\gcd(144,89)$ as a linear combination of 144 and 89:
\\ \rule{\linewidth}{0.5pt}
\[
\begin{aligned}
144 &= 1\cdot89 + 55, \\
89  &= 1\cdot55 + 34, \\
55  &= 1\cdot34 + 21, \\
34  &= 1\cdot21 + 13, \\
21  &= 1\cdot13 + 8, \\
13  &= 1\cdot8 + 5, \\
8   &= 1\cdot5 + 3, \\
5   &= 1\cdot3 + 2, \\
3   &= 1\cdot2 + 1, \\
2   &= 2\cdot1 + 0.
\end{aligned}
\]
Thus, $\gcd(144,89) = 1$.
Back-substitution (working upward) yields the correct linear combination
\[
1 = 34\cdot144 - 55\cdot89,
\]
so
\[
\gcd(144,89) = 1 = 144(34) + 89(-55).
\]

\newpage
\noindent\underline{45.\;Describe the extended Euclidean algorithm using pseudocode:}
\[ax + by = \gcd(a,b).\]

\noindent\textbf{Pseudocode:}
\begin{enumerate}
    \item \textbf{Input:} Two integers $a, b$.
    \item \textbf{Output:} $\gcd(a,b)$ and integers $x, y$ such that $ax + by = \gcd(a,b)$.
    \item If $b = 0$, then:
    \begin{enumerate}
        \item return $(a, 1, 0)$.
    \end{enumerate}
    \item Otherwise:
    \begin{enumerate}
        \item Recursively call the algorithm with $(b, a \bmod b)$.
        \item Suppose it returns $(d, x_1, y_1)$ such that $bx_1 + (a \bmod b) y_1 = d$.
        \item Then set:
        \[
        x = y_1, \quad y = x_1 - \left\lfloor \tfrac{a}{b} \right\rfloor y_1.
        \]
        \item Return $(d, x, y)$.
    \end{enumerate}
\end{enumerate}

\noindent\textbf{Example with $a=123$, $b=2347$:}

\begin{enumerate}
    \item Call $\text{EEA}(123,2347)$.
    \item Since $b \neq 0$, compute $123 \bmod 2347 = 123$ and recurse on $(2347,123)$.
    \item Call $\text{EEA}(2347,123)$. Compute:
    \[
    2347 = 123\cdot 19 + 10,\quad 123 = 10\cdot 12 + 3,\quad 10 = 3\cdot3 + 1,\quad 3 = 1\cdot3 + 0.
    \]
    \item Back-substitute and compute the coefficients. One correct identity is:
    \[
    123(-706) + 2347(37) = 1.
    \]
    \item Therefore $\gcd(123,2347) = 1$ and one solution is $x=-706$, $y=37$.
\end{enumerate}

\newpage
\subsection*{week 2}
\addcontentsline{toc}{subsection}{week 2}
\subsubsection*{Section 4.4 (Solving Congruences)}
\addcontentsline{toc}{subsubsection}{Section 4.4 (Solving Congruences)}
Questions: 5, 9, 11, 15, 21, 22, 23, 27
\\\\
\noindent\underline{5.\;Find an inverse of $a$ modulo $m$ of each of these pairs:}
\begin{enumerate}[label=(\alph*)]
    \item $a =4, m=9$
    \[
    \text{must find } x \text{ such that } 4x \equiv 1 \pmod{9}.
    \]
    Using the extended Euclidean algorithm:
    \begin{align*}
    9 &= 2 \cdot 4 + 1, \\
    4 &= 4 \cdot 1 + 0.
    \end{align*}
    Back-substituting gives:
    \[
    1 = 9 - 2 \cdot 4.
    \]
    Rearranging to isolate the coefficient of 4 gives $1 = (-2)\cdot4 + 1\cdot9$, so one inverse is $x=-2\equiv7\pmod9$.
    \[
    \therefore 4^{-1} \equiv 7 \pmod{9}.
    \]

    \item $a =19, m=141$
    Using the extended Euclidean algorithm (steps omitted for brevity but can be computed), one finds
    \[
    1 = 19(52) + 141(-7),
    \]
    so $19^{-1} \equiv 52 \pmod{141}$.

    \item $a =55, m=89$
    Using the extended Euclidean algorithm one finds
    \[
    1 = 55(34) + 89(-21),
    \]
    therefore $55^{-1} \equiv 34 \pmod{89}$.

    \item $a =89, m=232$
    Using the extended Euclidean algorithm one finds
    \[
    1 = 89(73) + 232(-28),
    \]
    therefore $89^{-1} \equiv 73 \pmod{232}$.
\end{enumerate}
\vspace{5mm}
\noindent\underline{9.\;Solve the congruence $4x \equiv 5 \pmod{9}$:}
\\
From question 5(a), $4^{-1} \equiv 7 \pmod{9}$.
Multiplying both sides of the congruence by $4^{-1}$ gives:
\[x \equiv 5 \cdot 7 \pmod{9}\]
\[x \equiv 35 \pmod{9}\]
Reducing $35$ modulo $9$ gives:
\[x \equiv 8 \pmod{9}\]

\newpage
\noindent\underline{11.\;Solve each of these congruences:}
\begin{enumerate}[label=(\alph*)]
    \item $19x \equiv 4 \pmod{141}$\\
From question 5(b), $19^{-1} \equiv 52 \pmod{141}$.
    \[
    x \equiv 4 \cdot 52 \equiv 208 \equiv 67 \pmod{141}.
    \]
    \item $55x \equiv 34 \pmod{89}$\\
From question 5(c), $55^{-1} \equiv 34 \pmod{89}$.
    \[
    x \equiv 34 \cdot 34 \equiv 1156 \equiv 88 \pmod{89}.
    \]
    \item $89x \equiv 2 \pmod{232}$\\
From question 5(d), $89^{-1} \equiv 73 \pmod{232}$.
    \[
    x \equiv 2 \cdot 73 \equiv 146 \pmod{232}.
    \]
\end{enumerate}

\vspace{5mm}
\noindent\underline{15.\;Show that if $m$ is an integer greater than $1$ and $ac \equiv bc \pmod{m}$, then $a \equiv b \pmod{\frac{m}{\gcd(c,m)}}$:}
\\\indent Suppose $ac \equiv bc \pmod{m}$. Then $m \mid c(a-b)$. 
Let $d=\gcd(c,m)$ and write $c = d c_1$ and $m = d m_1$, so that $\gcd(c_1,m_1)=1$. Since $m \mid c(a-b)$, we have $d m_1 \mid d c_1 (a-b)$. Cancelling the common factor $d$ gives $m_1 \mid c_1 (a-b)$. Because $\gcd(c_1,m_1)=1$, it follows that $m_1 \mid (a-b)$. Therefore $a \equiv b \pmod{m_1}$. Finally, since $m_1 = \tfrac{m}{d}$, we conclude that
\[
a \equiv b \pmod{\tfrac{m}{\gcd(c,m)}}.
\]

\vspace{5mm}
\noindent\underline{21.}Use the construction in the proof of the Chinese Remainder Theorem to find all solutions to the system:
\[
 x \equiv 1 \pmod{2},\quad
 x \equiv 2 \pmod{3},\quad
 x \equiv 3 \pmod{5},\quad
 x \equiv 4 \pmod{11}.
\]
\\ \rule{\linewidth}{0.5pt}
Let $m_1=2$, $m_2=3$, $m_3=5$, and $m_4=11$. Then $M=m_1 m_2 m_3 m_4 = 330$. We compute:
\[ M_1 = \frac{M}{m_1} = 165, \quad M_2 = \frac{M}{m_2} = 110, \quad M_3 = \frac{M}{m_3} = 66, \quad M_4 = \frac{M}{m_4} = 30. \]
Next, we find the inverses:
\[ y_1 \text{ such that } 165y_1 \equiv 1 \pmod{2} \implies y_1 = 1, \]
\[ y_2 \text{ such that } 110y_2 \equiv 1 \pmod{3} \implies y_2 = 2, \]
\[ y_3 \text{ such that } 66y_3 \equiv 1 \pmod{5} \implies y_3 = 1, \]
\[ y_4 \text{ such that } 30y_4 \equiv 1 \pmod{11} \implies y_4 = 7. \]
Now, we can construct the solution:
\[
x \equiv a_1 M_1 y_1 + a_2 M_2 y_2 + a_3 M_3 y_3 + a_4 M_4 y_4 \pmod{M}.
\]
Substituting the values we found:
\[
x \equiv 1 \cdot 165 \cdot 1 + 2 \cdot 110 \cdot 2 + 3 \cdot 66 \cdot 1 + 4 \cdot 30 \cdot 7 \pmod{330}.
\]
Calculating each term:
\[
x \equiv 165 + 440 + 198 + 840 \equiv 1643 \pmod{330}.
\]
Reducing $1643$ modulo $330$ gives:
\[
x \equiv 323 \pmod{330}.
\]
Thus, the solution to the system is:
\[
x \equiv 323 \pmod{330}.
\]

\newpage
\noindent\underline{22.}Solve the system of congruences $x \equiv 3 \pmod{6}$ and $x \equiv 4 \pmod{7}$ using back substitution:
\\ \rule{\linewidth}{0.5pt}
From the first congruence, $x=6k+3$. Substitute into the second:
\[
6k+3 \equiv 4 \pmod{7} \implies 6k \equiv 1 \pmod{7}.
\]
Since $6^{-1}\equiv6\pmod7$, $k\equiv6\pmod7$. So $k=7m+6$ and
\[
x = 6(7m+6)+3 = 42m + 39.
\]
Therefore $x \equiv 39 \pmod{42}$.

\vspace{5mm}

\newpage
\noindent\underline{27.\;Find all solutions (if any) to the system:}
\[
 x \equiv 7 \pmod{9},\quad
 x \equiv 4 \pmod{12},\quad
 x \equiv 16 \pmod{21}.
\]
\\ \rule{\linewidth}{0.5pt}
From the first congruence, $x = 9k + 7$. Substitute into the second:
\[
9k + 7 \equiv 4 \pmod{12} \implies 9k \equiv 9 \pmod{12}.
\]
Since $\gcd(9,12)=3$, divide through by 3 to get $3k \equiv 3 \pmod{4}$, i.e. $k \equiv 1 \pmod{4}$. So $k=1+4t$ and
\[
x = 9(1+4t)+7 = 16 + 36t.
\]
Substitute into the third congruence:
\[
16 + 36t \equiv 16 \pmod{21} \implies 36t \equiv 0 \pmod{21}.
\]
Since $\gcd(36,21)=3$, divide by 3: $12t \equiv 0 \pmod{7}$. Because $12\equiv5\pmod7$ and $\gcd(5,7)=1$, we get $t\equiv 0 \pmod{7}$, so $t=7s$.
Thus
\[
x = 16 + 36\cdot 7s = 16 + 252s,\quad s\in\mathbb{Z}.
\]
Therefore $x \equiv 16 \pmod{252}$.

\newpage

\subsection*{Week 3}
\addcontentsline{toc}{subsection}{Week 3}
\subsubsection*{Section 4.1 (Divisibility and Modular Arithmetic)}
\addcontentsline{toc}{subsubsection}{Section 4.1 (Divisibility and Modular Arithmetic)}
Questions: 48, 49, 50
\\\\


\noindent\underline{48.}Show that $\mathbb{Z}_m$ with addition modulo $m$, where $m \geq 2$ is an integer, satisfies closure, associativity, and commutativity; $0$ is an additive identity; and for every $a\in\mathbb{Z}_m$ the element $m-a$ is an additive inverse of $a$:
\\\rule{\linewidth}{0.5pt}

\textbf{Closure:} For any $a, b \in \mathbb{Z}_m$, the sum $a + b$ (taken modulo $m$) is again an element of $\mathbb{Z}_m$. Thus $\mathbb{Z}_m$ is closed under addition modulo $m$.

\textbf{Associativity:} Integer addition is associative, and reducing modulo $m$ preserves associativity, so for any $a,b,c\in\mathbb{Z}_m$,
\[
(a+b)+c \equiv a+(b+c) \pmod{m}.
\]

\textbf{Commutativity:} Integer addition is commutative, hence so is addition modulo $m$:
\[
a+b \equiv b+a \pmod{m}.
\]

\textbf{Additive identity:} The class $0\in\mathbb{Z}_m$ satisfies $a+0\equiv a\pmod m$ for all $a\in\mathbb{Z}_m$.

\textbf{Additive inverses:} For any $a\in\mathbb{Z}_m$, the element $m-a$ (interpreted modulo $m$) satisfies $a+(m-a)\equiv 0\pmod m$.

Therefore $\mathbb{Z}_m$ is an abelian group under addition modulo $m$.

\newpage
\noindent\underline{49.}Show that $\mathbb{Z}_m$ with multiplication modulo $m$, where $m \geq 2$ is an integer, satisfies closure, associativity, and commutativity, and $1$ is a multiplicative identity. Is it a group?
\\\rule{\linewidth}{0.5pt}
\textbf{Closure:} For $a,b\in\mathbb{Z}_m$, the product $a\cdot b \pmod m$ lies in $\mathbb{Z}_m$.

\textbf{Associativity:} Integer multiplication is associative, and reduction modulo $m$ preserves associativity.

\textbf{Commutativity:} Integer multiplication is commutative, so multiplication modulo $m$ is commutative.

\textbf{Multiplicative identity:} $1\in\mathbb{Z}_m$ satisfies $a\cdot 1 \equiv a \pmod m$ for all $a\in\mathbb{Z}_m$.

\textbf{Is it a group?} Not necessarily. Not every element of $\mathbb{Z}_m$ has a multiplicative inverse modulo $m$. For example, when $m=4$, the element $2$ has no inverse modulo $4$. Thus $\mathbb{Z}_m$ under multiplication is a commutative monoid with identity $1$, but it is a group only when every nonzero element is invertible — which happens exactly when $m$ is prime. The subgroup of units
\[
\mathbb{Z}_m^\times = \{[a]_m : \gcd(a,m)=1\}
\]
is an abelian group under multiplication modulo $m$.

\newpage

\noindent\underline{50.}Show that the distributive property of multiplication over addition holds for $\mathbb{Z}_m$, where $m \geq 2$ is an integer.
\\\rule{\linewidth}{0.5pt}

For any $a, b, c \in \mathbb{Z}_m$, integer arithmetic satisfies $a(b+c) = ab + ac$. Reducing both sides modulo $m$ preserves equality, so
\[
a\cdot (b + c) \equiv a\cdot b + a\cdot c \pmod{m}.
\]
Thus, the distributive law holds in $\mathbb{Z}_m$.

\end{document}
